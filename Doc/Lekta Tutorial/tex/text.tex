\documentclass[a4paper,10pt]{article}

\usepackage[dvips]{graphicx}
\usepackage[a4paper,left=3cm,right=3cm,top=4cm,bottom=4cm]{geometry}
\usepackage{amssymb,amsmath,amsthm,stmaryrd}
\usepackage{indentfirst}
\usepackage{latexsym}
\usepackage[dvipsnames]{xcolor}
\usepackage{fancybox}
\usepackage{tikz}
\usepackage{listings}
\usepackage{inconsolata} 

\usetikzlibrary{arrows,shapes,positioning}

\definecolor{red}{rgb}{0.5,0,0} 			% strings
\definecolor{green}{rgb}{0.05,0.4,0.25} 	% comments
\definecolor{purple}{rgb}{0.5,0,0.35} 		% keywords
\definecolor{blue}{rgb}{0.25,0.35,0.75} 	% types
\definecolor{gray}{rgb}{0.5,0.5,0.5} 		% delimiters
\definecolor{brown}{rgb}{0.5,0.25,0} 		% messages

\lstdefinelanguage{lekta}
{
	stringstyle = \color{red},
	commentstyle = \color{green},
	morecomment=[l]{//},
	morecomment=[s]{/*}{*/},
	sensitive = true,
	string = [b]",
	morestring = [b]',
	tabsize = 3,
	morekeywords = [1] { classDef,StructureComplex},
	morekeywords = [2] {ElementBool,bool},	
	keywordstyle = [1] \color{purple},
	keywordstyle = [2] \color{blue}\bfseries,
	showspaces = true,
}

\lstdefinestyle{lekta-plain}
{
    language = lekta,
    basicstyle = \scriptsize\ttfamily\mdseries\fontfamily{zi4}\selectfont
}

\lstset
{
    language = lekta,
    style = lekta-plain,
    showstringspaces = false,
    breaklines
}

\renewcommand*{\ttdefault}{cmtt}

\begin{document}
\large
\begin{center}
\textsc{Lekta documents} \\

\LARGE
\vspace{24pt}
\textbf{Lekta Tutorial}
\vspace{24pt}
\end{center}

\normalsize
\begin{flushleft}
\emph{Document Type:} Tutorial \\
\emph{Date:} 2016/11/29 \\
\emph{Version:} 0.1
\end{flushleft}

\vspace{24pt}
\section{Introduction}

\subsection{What is Natural Language Processing?}


\subsection{What is Lekta?}

Lekta is a high level programming language specially designed to create and execute applications 
related with Natural Language Processing (NLP).

\begin{lstlisting}[language=lekta]

classDef:StructureComplex 
(
	DialogueState : 
	(
		DialoguePhase,
		ActivatedScriptsStack,
		CurrentScriptDescriptor,
		NotDigestedProferenceChunks,
		PulledScript
	)
)

classDef:ElementBool
(
	PulledScript
)

classDef:Synonym
(
	NotDigestedProferenceChunks = ProferenceChunks
)

// This is a comment
/* This is a multi-line
	comment */
	
bool parameterTypesContains(ParameterTypes parameterTypesBatch, ParameterType parameterType)
{
	string s1 <- 'Hello, world';
	string s2 <- "Hello, world!";
	ParameterType recoveredParameterType;

	int size <- BatchSize( parameterTypesBatch );
	for(int position <- 1; position <= size; position++)
	{
		BatchRecoverPosition(parameterTypesBatch, position, recoveredParameterType);
		if( recoveredParameterType == parameterType )
		{
			return True;
		}
	}
	
	return False;
}
\end{lstlisting}

\end{document}



%\begin{enumerate}
%	\item Introduction
%	\begin{enumerate}
%		\item What is Lekta?
%		\item Features of Lekta Programming Language
%	\end{enumerate}
%	\item Your first Lekta project
%		\begin{enumerate}
%			\item Setting up a lekta project
%			\item Folder and file structure
%		\end{enumerate}
%	\item Automatic Speech Recognition
%	\item Natural Language Understanding
%	\begin{enumerate}
%		\item Lexicons
%		\item Grammars
%	\end{enumerate}
%	\item Dialogue Manager
%	\begin{enumerate}
%		\item Colligo
%		\item Senso
%		\item MindBoard
%		\item Respondo
%		\item Locutio
%	\end{enumerate}
%	\item Natural Language Generation
%	\begin{enumerate}
%		\item Templates
%	\end{enumerate}
%	\item Text to Speech Synthesis
%	\item Programmer Reference
%	\begin{enumerate}
%		\item Metatypes
%		\item Declaring new types in Lekta
%		\item Basic programming structures
%		\item Lekta own commands 
%		\item Built-in functions
%	\end{enumerate}
%\end{enumerate}
