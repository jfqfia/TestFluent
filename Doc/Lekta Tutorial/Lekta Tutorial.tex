
\documentclass[a4paper,10pt]{article}

\usepackage[dvips]{graphicx}
\usepackage[a4paper,left=3cm,right=3cm,top=4cm,bottom=4cm]{geometry}
\usepackage{amssymb,amsmath,amsthm,stmaryrd}
\usepackage{indentfirst}
\usepackage{latexsym}
\usepackage[dvipsnames]{xcolor}
\usepackage{fancybox}
\usepackage{tikz}

\usetikzlibrary{arrows,shapes,positioning}

\newcommand{\vname}[1]{\textcolor{OliveGreen}{\texttt{#1}}}
\newcommand{\typename}[1]{\textcolor{Blue}{\texttt{#1}}}

\begin{document}
\large
\begin{center}
\textsc{Lekta documents} \\

\LARGE
\vspace{24pt}
\textbf{Lekta Tutorial and Documentation}
\vspace{24pt}
\end{center}

\normalsize
\begin{flushleft}
\emph{Document Type:} Tutorial \\
\emph{Date:} 2016/09/26 \\
\emph{Version:} 0.1
\end{flushleft}

\vspace{24pt}
\section{Possible index of Lekta tutorial}

\begin{enumerate}
	\item Introduction
	\begin{enumerate}
		\item What is Lekta?
		\item Features of Lekta Programming Language
	\end{enumerate}
	\item Your first Lekta project
		\begin{enumerate}
			\item Setting up a lekta project
			\item Folder and file structure
		\end{enumerate}
	\item Automatic Speech Recognition
	\item Natural Language Understanding
	\begin{enumerate}
		\item Lexicons
		\item Grammars
	\end{enumerate}
	\item Dialogue Manager
	\begin{enumerate}
		\item Colligo
		\item Senso
		\item MindBoard
		\item Respondo
		\item Locutio
	\end{enumerate}
	\item Natural Language Generation
	\begin{enumerate}
		\item Templates
	\end{enumerate}
	\item Text to Speech Synthesis
	\item Programmer Reference
	\begin{enumerate}
		\item Metatypes
		\item Declaring new types in Lekta
		\item Basic programming structures
		\item Lekta own commands 
		\item Built-in functions
	\end{enumerate}
\end{enumerate}

\end{document}
